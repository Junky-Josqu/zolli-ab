\documentclass[11pt,twoside,a4paper]{exam}
\usepackage[small]{cwpuzzle}
\usepackage{graphicx}
\usepackage[ngerman]{babel}
\usepackage[left=1.5cm, right=1.5cm, top=1.5cm, bottom=1.5cm]{geometry}
\begin{document}

\begin{center}
  \huge{Arbeitsblatt}
\end{center}
\begin{center}
\fbox{\fbox{\parbox{5.5in}{\centering
Beantworte die folgenden fragen richtig. Wenn der Platzt ausgeht nutzte dein Heft. Wenn Antworten ungewiss nutzte dein Handout.}}}
\end{center}
\vspace{0.1in}
\makebox[\textwidth]{Name und Klasse:\enspace\hrulefill}

\begin{questions}
\question Fuelle die Defintion aus. wenn du fehler machst schreib sie richtig ab.

Glutenunvertraeglichkeit auch: \fillin ist eine chronische \fillin die sowohl Merkmale einer Allergie als auch einer \fillin aufweist.
Der Erkrankte reagier \fillin auf den Kleber \fillin mit einer chronisches \fillin der \fillin. Linderung verschaft
nur eine \fillin.



\question
  Kreuze die richtigen Symptome an.
  \begin{checkboxes}
\choice Chronische Entzündung der Dünndarmschleimhaut
\choice Adipositas
\choice Entwicklungsstörungen
\choice Augenschmerzen
\choice Durchfall/Erbrechen
  \end{checkboxes}



  \question Sortiere die schritte der Diagnose mithilfe von Zahlen.
  
  \begin{oneparcheckboxes}
    \choice{Dünndarmbiopsie}
    \choice{Verdacht}
    \choice{Bluttest}
  \end{oneparcheckboxes}


\begin{center}
\fbox{\fbox{\parbox{5.5in}{\centering
Gluten muss in der Zutatenliste von Lebensmittel gekennzeichnet werden. Folgende Zutaten enthalten Gluten, wenn nicht anders gekennzeichnet: 

 Gluten, (Alles mit) Weizen, Gerste(-malz), Roggen, Hafer, Dinkel, Einkorn, Kamut, Seitan, Kann spuren von Gluten...

}}}
\end{center}

\vspace{0.1in}

  \question Der Erkrankte darf Gluten nicht essen. Du siehst 3 fiktive Zutatenlisten A, B, und C kreuze an welchne fuer den Erkrankten geniessbar sind.
     \begin{oneparcheckboxes}
     ┊ \choice{A}
     ┊ \choice{B}
  ┊    \choice{C}
     \end{oneparcheckboxes}


\begin{minipage}[t]{0.3\textwidth}
 A:

 Maisflockeb, Reissirup, Kokosfett
\end{minipage}
\begin{minipage}[t]{0.3\textwidth}
 B: 

 Schokomilch, Pinguinfett, Seitan
\end{minipage}
\begin{minipage}[t]{0.3\textwidth}
 C:

 Walfett, Schweinezeh, Weizenmehl
\end{minipage}


\vspace{0.4in}
\begin{Puzzle}{15}{15}
  |{}  |{}  |{}  |{}  |[4]G|L   |U   |T   |E   |N   |I   |N   |{}  |[1]G|.
  |{}  |{}  |{}  |{}  |E   |{}  |{}  |{}  |{}  |{}  |{}  |{}  |{}  |L   |.
  |{}  |[5]G|[6]L|U   |T   |E   |N   |{}  |{}  |{}  |{}  |{}  |{}  |I   |.
  |{}  |{}  |y   |{}  |R   |{}  |{}  |{}  |{}  |{}  |{}  |{}  |{}  |A   |.
  |{}  |{}  |m   |{}  |E   |{}  |{}  |{}  |{}  |{}  |{}  |{}  |{}  |D   |.
  |{}  |{}  |p   |{}  |I   |{}  |{}  |{}  |{}  |{}  |{}  |{}  |{}  |I   |.
  |{}  |{}  |h   |{}  |[2]D|A   |R   |M   |Z   |O   |[3]T|T   |E   |N   |.
  |{}  |{}  |o   |{}  |E   |{}  |{}  |{}  |{}  |{}  |G   |{}  |{}  |{}  |.
  |{}  |{}  |z   |{}  |{}  |{}  |{}  |{}  |{}  |{}  |2   |{}  |{}  |{}  |.
  |{}  |{}  |y   |{}  |{}  |{}  |{}  |{}  |{}  |{}  |{}  |{}  |{}  |{}  |.
  |{}  |{}  |t   |{}  |{}  |{}  |{}  |{}  |{}  |{}  |{}  |{}  |{}  |{}  |.
  |{}  |{}  |e   |{}  |{}  |{}  |{}  |{}  |{}  |{}  |{}  |{}  |{}  |{}  |.
  |{}  |{}  |n   |{}  |{}  |{}  |{}  |{}  |{}  |{}  |{}  |{}  |{}  |{}  |.
\end{Puzzle}
\begin{PuzzleClues}{\textbf{Wagerecht:}}
  \Clue{2}{DARMZOTTEN}{Beim Erkrankten schrumpfen jene.}
  \Clue{4}{GLUTENIN}{Ein anderes Proteingemisch.}
  \Clue{5}{GLUTEN}{Ein Stoffgemisch aus 2 Proteingemischen auf das der Erkrankte negative reagiert.}
\end{PuzzleClues}
\begin{PuzzleClues}{\textbf{Senkrecht:}}
  \Clue{1}{GLIADIN}{Ein Proteingemisch.}
  \Clue{3}{TG2}{Enzym das Proteingemische spaltet}
  \Clue{4}{GETREIDE}{Erkrankte reagieren auf bestimmtes...}
  \Clue{6}{Lymphozyten}{Teil vom Immunsystem, verursacht Entzündungen.}
\end{PuzzleClues}
\end{questions}

\end{document}
