\documentclass[11pt,twoside,a4paper]{exam}
\usepackage[small]{cwpuzzle}
\usepackage{graphicx}
\usepackage[ngerman]{babel}
\usepackage[left=1.5cm, right=1.5cm, top=1.5cm, bottom=1.5cm]{geometry}
\begin{document}

\begin{center}
  \huge{Arbeitsblatt}
\end{center}
\begin{center}
\fbox{\fbox{\parbox{5.5in}{\centering
Beantworte die folgenden fragen richtig. Wenn der Platzt ausgeht nutzte dein Heft. Wenn Antworten ungewiss nutzte dein Handout.}}}





\end{center}
\vspace{0.1in}
\makebox[\textwidth]{Name und Klasse:\enspace\hrulefill}

\begin{questions}
\question Fuelle die Defintion aus.  wenn du fehler machst schreibe sie danach richtig ab ins Heft.

Glutenunvertraeglichkeit auch: \fillin(lat.) ist eine chronische \fillin die sowohl Merkmale einer Allergie als auch einer \fillin aufweist.
Der Erkrankte reagier \fillin auf den Kleber \fillin mit einer chronisches \fillin der \fillin. Linderung verschaft
nur eine \fillin.



\question
  Kreuze die richtigen Symptome an.
  \begin{checkboxes}
\choice Chronische Entzündung der Dünndarmschleimhaut
\choice Adipositas
\choice Entwicklungsstörungen
\choice Augenschmerzen
\choice Durchfall/Erbrechen
  \end{checkboxes}



  \question Sortiere die schritte der Diagnose mithilfe von den Zahlen 1-3.
  
  \begin{oneparcheckboxes}
    \choice{Dünndarmbiopsie}
    \choice{Verdacht}
    \choice{Bluttest}
  \end{oneparcheckboxes}

 Kreuze richtig An. Korrigiere ggf.
\begin{questions}
  \question TG2 ist Hormon.
    \begin{oneparcheckboxes}
    \choice{Korrekt}
    \choice{Falsch}
  \end{oneparcheckboxes}
   \question Glutenin ist ein Enzym.
    \begin{oneparcheckboxes}
    \choice{Korrekt}
    \choice{Falsch}
  \end{oneparcheckboxes}
  \question Pluto ist ein Planet!
   \begin{oneparcheckboxes}
    \choice{Korrekt}
    \choice{Falsch}
  \end{oneparcheckboxes}
  \question Gluten ist giftig fuer jeden.
   \begin{oneparcheckboxes}
    \choice{Korrekt}
    \choice{Falsch}
  \end{oneparcheckboxes}
  \question Gliadin ist ein Spaltprodukt.
      \begin{oneparcheckboxes}
    \choice{Korrekt}
    \choice{Falsch}
  \end{oneparcheckboxes}
\end{questions}

\question Gluten im Gesunden Koerper. Sortiere die Rheinfolge mithile von den Zahlen 1-3. Schreib ab wenn unsicher
  \begin{checkboxes}
    \choice{Duenndarmschleimhaut nimmt Spaltprodukte auf.}
    \choice{TG2 spaltet Gluten in Gliadin und Glutening.}
    \choice{Orale Aufnahme von Gluten.}
  \end{checkboxes}

\question Gluten im kranken Koerper. Sortiere die Rheinfolge mithile von den Zahlen 1-3. Schreib ab wenn unsicher
  \begin{checkboxes}
    \choice{Alles wie im Gesunden Koerper}
    \choice{Koerpereigenes Enzym TG2 und fremdes Glutenin wird vom Immunsystem angegriffen}
    \choice{Lympozyten bilden sich und sorgen fuer eine Entzuendund.}
  \end{checkboxes}

\begin{center}
\fbox{\fbox{\parbox{5.5in}{\centering
Eine Allergie ist grob gesagt wenn jemand auf ein koerperfremdes objekt eine immunreaktion verursacht obwohl keine Gefahr besteht.
}}}
\end{center}

\begin{center}
\fbox{\fbox{\parbox{5.5in}{\centering
Eine Autoimmungerkranung ist grob gesagt wenn jemand auf ein koerpereigenes objekt eine immungreaktion verursacht obwohl keine Gefahr besteht.
}}}
\end{center}

\question Nutzte die zwei kasten oben und kreuze an was Glutenunvertraeglickeit ist.
  
\begin{oneparcheckboxes}
    \choice{Es ist eine Allergie}
    \choice{Es ist eine Autoimmunerkrankung}
    \choice{Sowohl als Auch}
  \end{oneparcheckboxes}

\begin{center}
\fbox{\fbox{\parbox{5.5in}{\centering
Gluten muss in der Zutatenliste von Lebensmittel gekennzeichnet werden. Folgende Zutaten enthalten Gluten, wenn nicht anders gekennzeichnet: 

 -Gluten, -(Alles mit) Weizen, -Gerste(-malz), -Roggen, -Hafer, -Dinkel, -Einkorn, -Kamut, -Seitan, -Kann spuren von Gluten enthalten.

Schreibe die Zutaten ab ins Heft.
}}}
\end{center}

\vspace{0.1in}

  \question Der Erkrankte darf Gluten nicht essen. Du siehst 3 fiktive Zutatenlisten A, B, und C kreuze an welchne fuer den Erkrankten geniessbar sind.
     \begin{oneparcheckboxes}
     ┊ \choice{A}
     ┊ \choice{B}
  ┊    \choice{C}
     \end{oneparcheckboxes}


\begin{minipage}[t]{0.3\textwidth}
 A:

 Maisflockeb, Reissirup, Kokosfett
\end{minipage}
\begin{minipage}[t]{0.3\textwidth}
 B: 

 Schokomilch, Pinguinfett, Seitan
\end{minipage}
\begin{minipage}[t]{0.3\textwidth}
 C:

 Walfett, Schweinezeh, Weizenmehl
\end{minipage}


\vspace{0.4in}
\begin{Puzzle}{15}{15}
  |{}  |{}  |{}  |{}  |[4]G|L   |U   |T   |E   |N   |I   |N   |{}  |[1]G|.
  |{}  |{}  |{}  |{}  |E   |{}  |{}  |{}  |{}  |{}  |{}  |{}  |{}  |L   |.
  |{}  |[5]G|[6]L|U   |T   |E   |N   |{}  |{}  |{}  |{}  |{}  |{}  |I   |.
  |{}  |{}  |y   |{}  |R   |{}  |{}  |{}  |{}  |{}  |{}  |{}  |{}  |A   |.
  |{}  |{}  |m   |{}  |E   |{}  |{}  |{}  |{}  |{}  |{}  |{}  |{}  |D   |.
  |{}  |{}  |p   |{}  |I   |{}  |{}  |{}  |{}  |{}  |{}  |{}  |{}  |I   |.
  |{}  |{}  |h   |{}  |[2]D|A   |R   |M   |Z   |O   |[3]T|T   |E   |N   |.
  |{}  |{}  |o   |{}  |E   |{}  |{}  |{}  |{}  |{}  |G   |{}  |{}  |{}  |.
  |{}  |{}  |z   |{}  |{}  |{}  |{}  |{}  |{}  |{}  |2   |{}  |{}  |{}  |.
  |{}  |{}  |y   |{}  |{}  |{}  |{}  |{}  |{}  |{}  |{}  |{}  |{}  |{}  |.
  |{}  |{}  |t   |{}  |{}  |{}  |{}  |{}  |{}  |{}  |{}  |{}  |{}  |{}  |.
  |{}  |{}  |e   |{}  |{}  |{}  |{}  |{}  |{}  |{}  |{}  |{}  |{}  |{}  |.
  |{}  |{}  |n   |{}  |{}  |{}  |{}  |{}  |{}  |{}  |{}  |{}  |{}  |{}  |.
\end{Puzzle}
\begin{PuzzleClues}{\textbf{Wagerecht:}}
  \Clue{2}{DARMZOTTEN}{Beim Erkrankten schrumpfen jene.}
  \Clue{4}{GLUTENIN}{Ein anderes Proteingemisch.}
  \Clue{5}{GLUTEN}{Ein Stoffgemisch aus 2 Proteingemischen auf das der Erkrankte negative reagiert.}
\end{PuzzleClues}
\begin{PuzzleClues}{\textbf{Senkrecht:}}
  \Clue{1}{GLIADIN}{Ein Proteingemisch.}
  \Clue{3}{TG2}{Enzym das Proteingemische spaltet}
  \Clue{4}{GETREIDE}{Erkrankte reagieren auf bestimmtes...}
  \Clue{6}{Lymphozyten}{Teil vom Immunsystem, verursacht Entzündungen.}
\end{PuzzleClues}
\end{questions}

\end{document}
