\documentclass[11pt,twoside,a4paper]{exam}
\usepackage[small]{cwpuzzle}
\usepackage{graphicx}
\usepackage[ngerman]{babel}
\usepackage[left=1.5cm, right=1.5cm, top=1.5cm, bottom=1.5cm]{geometry}
\begin{document}

\begin{center}
  \huge{Z\"oliakie Arbeitsblatt}
\end{center}
\begin{center}
\fbox{\fbox{\parbox{5.5in}{\centering
Beantworte die folgenden Fragen richtig. Wenn der Platzt ausgeht nutzte dein Heft. Wenn Antworten ungewiss nutzte dein Handout.}}}
\end{center}
\vspace{0.1in}
\makebox[\textwidth]{Name und Klasse:\enspace\hrulefill}

\begin{questions}
\question F\"ulle die Defintion aus. Wenn du fehler machst, schreibe sie danach richtig ab ins Heft.

Glutenunvertr\"aglichkeit auch: \fillin(lat.) ist eine chronische \fillin die sowohl Merkmale einer Allergie als auch einer \fillin aufweist.
Der Erkrankte reagiert \fillin auf den Kleber \fillin mit einer chronischen \fillin der \fillin. Linderung verschaft
nur eine \fillin.



\question
  Kreuze die richtigen Symptome an.
  \begin{checkboxes}
\choice Chronische Entz\"undung der D\"unndarmschleimhaut
\choice Adipositas
\choice Entwicklungsstörungen
\choice Augenschmerzen
\choice Durchfall/Erbrechen
  \end{checkboxes}



  \question Sortiere die 3 Schritte der Diagnose mithilfe von Zahlen.
  
  \begin{oneparcheckboxes}
    \choice{D\"unndarmbiopsie}
    \choice{Verdacht}
    \choice{Bluttest}
  \end{oneparcheckboxes}


\question Kreuze richtig an!
\begin{questions}
  \question TG2 ist ein Hormon.
    \begin{oneparcheckboxes}
    \choice{Korrekt}
    \choice{Falsch}
  \end{oneparcheckboxes}
   \question Glutenin ist ein Enzym.
    \begin{oneparcheckboxes}
    \choice{Korrekt}
    \choice{Falsch}
  \end{oneparcheckboxes}
  \question Pluto ist ein Planet!
   \begin{oneparcheckboxes}
    \choice{Korrekt}
    \choice{Falsch}
  \end{oneparcheckboxes}
  \question Gluten ist giftig f\"ur jeden.
   \begin{oneparcheckboxes}
    \choice{Korrekt}
    \choice{Falsch}
  \end{oneparcheckboxes}
  \question Gliadin ist ein Spaltprodukt.
      \begin{oneparcheckboxes}
    \choice{Korrekt}
    \choice{Falsch}
  \end{oneparcheckboxes}
\end{questions}

\question Gluten im Gesunden K\"orper. Sortiere die Rheinfolge mithilfe von den Zahlen 1-3. Schreibe ab wenn unsicher.
  \begin{checkboxes}
    \choice{D\"unndarmschleimhaut nimmt Spaltprodukte auf.}
    \choice{TG2 spaltet Gluten in Gliadin und Glutening.}
    \choice{Orale Aufnahme von Gluten.}
  \end{checkboxes}

\question Gluten im kranken K\"orper. Sortiere die Rheinfolge mithilfe von den Zahlen 1-3. Schreibe ab wenn unsicher
  \begin{checkboxes}
    \choice{Alles wie im Gesunden K\"orper}
    \choice{K\"orpereigenes Enzym TG2 und fremdes Glutenin wird vom Immunsystem angegriffen}
    \choice{Lympozyten bilden sich und sorgen f\"ur eine Entz\"undund.}
  \end{checkboxes}

\begin{center}
\fbox{\fbox{\parbox{5.5in}{\centering
Eine Allergie ist grob gesagt, wenn jemand auf ein k\"orperfremdes Objekt eine Immunreaktion verursacht. Obwohl keine Gefahr besteht.
}}}
\end{center}

\begin{center}
\fbox{\fbox{\parbox{5.5in}{\centering
Eine Autoimmungerkranung ist grob gesagt, wenn jemand auf ein k\"orpereigenes Objekt eine Immunreaktion verursacht. Obwohl keine Gefahr besteht.
}}}
\end{center}

\question Nutzte die zwei kasten oben und kreuze an was Glutenunvertr\"aglickeit ist.
  
\begin{oneparcheckboxes}
    \choice{Es ist eine Allergie}
    \choice{Es ist eine Autoimmunerkrankung}
    \choice{Sowohl als Auch}
  \end{oneparcheckboxes}

\begin{center}
\fbox{\fbox{\parbox{5.5in}{\centering
Gluten(und Gluten-haltige Zutaten) m\"ussen in der Zutatenliste von Lebensmittel gelistet sein. Folgende Zutaten enthalten Gluten, wenn nicht anders gekennzeichnet: 

 -Gluten, -(Alles mit) Weizen, -Gerste(-malz), -Roggen, -Hafer, -Dinkel, -Einkorn, -Kamut, -Seitan, -Kann spuren von Gluten enthalten.

Schreibe die Zutaten ab, ins Heft.
}}}
\end{center}

\vspace{0.1in}

  \question Der Erkrankte darf Gluten nicht essen. Du siehst 3 fiktive Zutatenlisten: A, B, und C kreuze an welche f\"ur den Erkrankten geniessbar sind.
     \begin{oneparcheckboxes}
     ┊ \choice{A}
     ┊ \choice{B}
  ┊    \choice{C}
     \end{oneparcheckboxes}

\vspace{0.3 in}
\begin{minipage}[t]{0.3\textwidth}
 A:

 Maisflocken, Reissirup, Kokosfett
\end{minipage}
\begin{minipage}[t]{0.3\textwidth}
 B: 

 Schokomilch, Pinguinfett, Seitan
\end{minipage}
\begin{minipage}[t]{0.3\textwidth}
 C:

 Walfett, Riesenzeh, Weizenmehl
\end{minipage}

\vspace{0.5 in}
\question F\"ulle richtig aus.

\begin{Puzzle}{15}{15}
  |{}  |{}  |{}  |{}  |[4]G|L   |U   |T   |E   |N   |I   |N   |{}  |[1]G|.
  |{}  |{}  |{}  |{}  |E   |{}  |{}  |{}  |{}  |{}  |{}  |{}  |{}  |L   |.
  |{}  |[5]G|[6]L|U   |T   |E   |N   |{}  |{}  |{}  |{}  |{}  |{}  |I   |.
  |{}  |{}  |y   |{}  |R   |{}  |{}  |{}  |{}  |{}  |{}  |{}  |{}  |A   |.
  |{}  |{}  |m   |{}  |E   |{}  |{}  |{}  |{}  |{}  |{}  |{}  |{}  |D   |.
  |{}  |{}  |p   |{}  |I   |{}  |{}  |{}  |{}  |{}  |{}  |{}  |{}  |I   |.
  |{}  |{}  |h   |{}  |[2]D|A   |R   |M   |Z   |O   |[3]T|T   |E   |N   |.
  |{}  |{}  |o   |{}  |E   |{}  |{}  |{}  |{}  |{}  |G   |{}  |{}  |{}  |.
  |{}  |{}  |z   |{}  |{}  |{}  |{}  |{}  |{}  |{}  |2   |{}  |{}  |{}  |.
  |{}  |{}  |y   |{}  |{}  |{}  |{}  |{}  |{}  |{}  |{}  |{}  |{}  |{}  |.
  |{}  |{}  |t   |{}  |{}  |{}  |{}  |{}  |{}  |{}  |{}  |{}  |{}  |{}  |.
  |{}  |{}  |e   |{}  |{}  |{}  |{}  |{}  |{}  |{}  |{}  |{}  |{}  |{}  |.
  |{}  |{}  |n   |{}  |{}  |{}  |{}  |{}  |{}  |{}  |{}  |{}  |{}  |{}  |.
\end{Puzzle}
\begin{PuzzleClues}{\textbf{Wagerecht:}}
  \Clue{\textbf 2}{DARMZOTTEN}{Beim Erkrankten schrumpfen jene:}
  \Clue{\textbf 4}{GLUTENIN}{Ein anderes Proteingemisch.}
  \Clue{\textbf 5}{GLUTEN}{Ein Stoffgemisch aus 2 Proteingemischen. Auf das der Erkrankte negative reagiert.}
\end{PuzzleClues}
\begin{PuzzleClues}{\textbf{Senkrecht:}}
  \Clue{\textbf 1}{GLIADIN}{Ein Proteingemisch.}
  \Clue{\textbf 3}{TG2}{Enzym das Proteingemische spaltet.}
  \Clue{\textbf 4}{GETREIDE}{Erkrankte reagieren auf bestimmtes:}
  \Clue{\textbf 6}{Lymphozyten}{Teil vom Immunsystem, verursacht Entz\"undungen.}
\end{PuzzleClues}
\end{questions}

\end{document}
